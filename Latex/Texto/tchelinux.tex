%Preâmbulo

%Define o tipo de documento
\documentclass{book}


%Pacotes+++++++++++++++++++++++++++++++++++++++++++++++++
%%figuras e graficos
\usepackage{float}
\usepackage{graphicx}
%%Pt-Br
\usepackage[portuguese]{babel} 
\usepackage[T1]{fontenc}
\usepackage[utf8]{inputenc}
\usepackage{booktabs}
%Citações abnt
\usepackage[alf,abnt-emphasize=bf]{abntex2cite}
%%+++++++++++++++++++++++++++++++++++++++++++++++++++++++

\title{Um Exemplo de Texto em \LaTeX}
\author{Ricardo Peixoto Robaina}

%%criando uma "variavel"
\newcommand{\tl}{Tchelinux}


%Corpo
\begin{document}
	
	%%Gera o título do documento	
	\maketitle
	
	%%pula uma pagina
	\newpage
	
	%%Gera o sumário
	\tableofcontents
	
	\newpage
	
	
	%%cria uma nova seção no texto
	\section{Introdução}
	\label{Intro} %% uma referência para fazer o cruzamento 
	Este texto é um exemplo de como utilizar \LaTeX. A seção \ref{tchelinux} trata do evento \tl. A seção \ref{quem} mostra quem somos. A seção \ref{mascote} aprenta o mascote do evento \tl.
	
	%% \ref{label} gera o indice correto da referência passada como parâmetro
	
	%% \tl vai inserir a variavél criada (Tchelinuz) 
	
	
	%%Citação de obras cadastradas no arquivo bibliografia.bib
	Vou fazer uma citação indireta. \cite{scrumGuide} e \cite{UnitySite} sao exemplos de referencias.
	
	\section{\tl}
	\label{tchelinux}
	\tl, Software Livre em Ação!
	
	A paixão pelo Software Livre assim como o desejo de compartilhar conhecimentos fez com que um grupo de usuários                      se reunisse e trabalhasse junto na organização de eventos para a divulgação de ferramentas e da filosofia open source em instituições educacionais do Rio Grande do Sul. Aproveitando a ideia de voluntariado, o grupo decidiu que os eventos seriam sempre gratuitos, porém os participantes são encorajados a doar 2kg de alimentos não-perecíveis, que são encaminhados a instituições de caridade da cidade onde ocorre o evento.
	
	Desde Outubro de 2006 o \tl\ realizou 84 eventos em 20 cidades com o auxílio de milhares de voluntários, que ajudaram na organização e apresentando 1291 palestras foram assistidas por 10268 pessoas que doaram 16295kg de alimentos para 40 instituições de caridade. 
	
	Há muito trabalho a ser feito e todos são bem-vindos a participar do \tl\ e ajudar a divulgar o Software Livre e levar o conhecimento à todos cantos do Rio Grande do Sul! 
	
	\textbf{Contamos com você! }
	
	Termos em inglês, como \textit{software}, devem estar em itálico.	
	
	
	\subsection{Quem Somos}
	\label{quem}
	O \tl é um grupo de voluntários que trabalha divulgação do Software Livre no estado do Rio Grande do Sul, através da organização e realização de eventos gratuitos para estudantes, profissionais da área de Tecnologia da Informação e demais interessados.
	
	O grupo promove também a caridade, uma vez que participantes de seus eventos são encorajados a doarem alimentos não-perecíveis para serem encaminhados à instituções de caridade.
	
	Nossas principais iniciativas:
	%%Cria uma lista de itens
	\begin{itemize}
		\item Divulgar o Software Livre no estado do Rio Grande do Sul
		\item Compartilhar conhecimento técnico e experiências
		\item Suporte a novos usuários de Software Livre
		\item Incentivo à caridade
	\end{itemize}
	
	
	\section{Mascote}
	\label{mascote}
	
	%%Insere uma figura
	\begin{figure}[!htb]
		\centering
		\includegraphics{figura.png}
		\caption{Mascote tche Linz}
		\label{figura_tux}
	\end{figure}
	
	
	O símbolo do software foi escolhido pelo seu criador, Linus Torvalds, que um dia estava no zoológico e foi surpreendido pela mordida de um pinguim. Fato curioso e discutido até hoje. 
	
	
	Em 1996, muitos integrantes da lista de discussão "Linux-Kernel" estavam discutindo sobre a criação de um logotipo ou de um mascote que representasse o Linux. Muitas das sugestões eram paródias ao logotipo de um sistema operacional concorrente e muito conhecido (Windows). Outros eram monstros ou animais agressivos. 
	
	Linus Torvalds acabou entrando nesse debate ao afirmar em uma mensagem que gostava muito de pinguins. Isso foi o suficiente para dar fim à discussão. Depois disso, várias tentativas foram feitas numa espécie de concurso para que a imagem de um pinguim servisse aos propósitos do Linux, até que Larry Ewing sugeriu a figura de um "pinguim sustentando o mundo". 
	
	Em resposta, Linus Torvalds declarou que achava interessante que esse pinguim tivesse uma imagem simples: um pinguim "gordinho" e com expressão de satisfeito, como se tivesse acabado de comer uma porção de peixes.
	
	Torvalds também não achava atraente a ideia de algo agressivo, mas sim a ideia de um pinguim simpático, do tipo em que as crianças perguntem: "Mamãe, posso ter um desses também?". 
	
	Ainda, Torvalds também frisou que trabalhando dessa forma, as pessoas poderiam criar várias modificações desse pinguim. Isso realmente acontece.
	
	Kurumim Linux, o meu Tux preferido
	Este, diferente do original, é um filhote, mais magro, um menino em forma de pinguim, corajoso, e com características brasileiras. 
	
	Linus confessou ter contraído "Pinguinite", doença transmitida por mordidas de pinguim. O sintoma é gostar muito de pinguins. 
	
	Quando questionado sobre o por quê de pinguins, Linus Torvalds respondeu que não havia uma razão em especial, mas os achava engraçados e até citou que foi bicado por um "pinguim assassino" na Austrália e ficou impressionado como a bicada de um animal aparentemente tão inofensivo, podia ser tão dolorosa. 
	A \figurename~\ref{mascote} apresenta o uma imagem do mascote do \tl. O mascote do \tl\ é
	o Tux pilchado.	
	%\includegraphics{figura.png}
	
	
	
	
	\section{Tabelas}
	
	%%insere uma nova tabela (melhorar utilizar ferramenas para criar as tabelas) https://www.tablesgenerator.com/
	\begin{table}[h]
		\centering
		\caption{Um nome qualquer}
		\vspace{0.5cm}
		\begin{tabular}{r|lr}
			
			Posi{\c c}{\~a}o & Pa{\'i}s & IDH \\ % Note a separação de col. e a quebra de linhas
			\hline                               % para uma linha horizontal
			1 & Noruega        & .955 \\
			2 & Austr{\'a}lia  & .938 \\
			3 & EUA            & .937 \\
			4 & Holanda        & .921 \\
			5 & Alemanha       & .920            % não é preciso quebrar a última linha
			
		\end{tabular}
	\end{table}
	
	
	%%Muito fácil criar fórumlas em LaTeX
	\section{Formulas Matemáticas}
	
	$$ X_1 + X_2 + X_3 $$
			
	
	A Equação \ref{bask} apresenta a fórmula de Baskara.
	
	\begin{equation}
		x=\frac{-b\pm\sqrt{b^2-4ac}}{2a} 
		\label{bask}
	\end{equation}
	
	
	%% Bibliografia
	%%um programa legal para gerencias as bibliografias é o JabRef http://www.jabref.org/
	\bibliography{bibliografia}
	
	
	
\end{document}


%Final documento