\documentclass{beamer}

\usefonttheme{structurebold}

\usepackage[utf8]{inputenc}
\usepackage[T1]{fontenc}
\usepackage[portuguese]{babel} 



\title{Introdução ao \LaTeX}
\date{\today}
\author{Ricardo Peixoto Robaina \\ Tchelinux Bagé 2018}
\institute{Universidade Federal do Pampa}

%%\usetheme{metropolis}




\begin{document}
	
	\begin{frame}[noframenumbering]
		\titlepage
		\thispagestyle{empty}
	\end{frame}
	
	\begin{frame}{Sumário}
		\setbeamertemplate{section in toc}[sections numbered]
		\tableofcontents[hideallsubsections]
	\end{frame}


\begin{frame}{Sumário}
$$x=\frac{-b\pm\sqrt{b^2-4ac}}{2a}$$

\end{frame}



\section{Introdução}







\begin{frame}
	Ola mundos
	processador de texto
	
	\Tex Donal Knuth (1977
	)
	
	Leslie Lamport escreveu o \LaTeX
\end{frame}

\section{Instalação}

\begin{block}{Compilador}
	\begin{itemize}
		\item texlive
		\item miktex
	\end{itemize}
\end{block}

\begin{block}{Editor}
	\begin{itemize}
		\item texstudio
		\item miktex
	\end{itemize}
\end{block}

\section{Exemplo Documento}

\section{Exemplo Slide}

\section{Contato}

\begin{frame}
	ricardorobaina11@gmail.com
	
	github.com/robainaricardo
\end{frame}


\end{document}
